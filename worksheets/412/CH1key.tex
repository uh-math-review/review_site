% DOCUMENT FORMATING
\documentclass[12pt]{article}
\usepackage[margin=1in]{geometry}

% PACKAGES
\usepackage{amsmath} % For extended formatting
\usepackage{amssymb} % For math symbols
\usepackage{amsthm} % For proof environment
\usepackage{array} % For tables
\usepackage{enumerate} % For lists
\usepackage{extramarks} % For headers and footers
\usepackage{blindtext}
\usepackage{fancyhdr} % For custom headers
\usepackage{graphicx} % For inserting images
\usepackage{multicol} % For multiple columns
\usepackage{verbatim} % For displaying code
\usepackage{tkz-euclide}
\usepackage{pgfplots}
\newtheorem{theorem}{Theorem}[section]
\newtheorem*{theorem*}{Theorem}
\newtheorem{corollary}{Corollary}[theorem]
\newtheorem{lemma}[theorem]{Lemma}

% SET UP HEADER AND FOOTER
\pagestyle{fancy}
\lhead{\MyCourse} % Top left header
\chead{\MyTopicTitle} % Top center header
\rhead{\MyAssignment} % Top right header
\lfoot{\MyCampus} % Bottom left footer
\cfoot{} % Bottom center footer
\rfoot{\MySemester} % Bottom right footer
\renewcommand\headrulewidth{0.4pt} % Size of the header rule
\renewcommand\footrulewidth{0.4pt} % Size of the footer rule
% ----------
% TITLES AND NAMES 
% ----------

\newcommand{\MyCourse}{Math 412}
\newcommand{\MyTopicTitle}{Division Algorithm}
\newcommand{\MyAssignment}{Abstract Algebra}
\newcommand{\MySemester}{Fall 2020}
\newcommand{\MyCampus}{University of Hawaii at Manoa}

\begin{document}
\subsection*{Key Terms}
\begin{itemize}
    \item Well Ordering Axiom \newline 
    every nonempty subset of the set of positive integers contains smallest element 
    \item Prime \newline 
    an integer is considered prime if the only divisors are $\pm 1$ and $\pm$ itself
    \item Relatively Prime \newline 
    two integers whose GCD is 1 
\end{itemize}
\subsection*{Theorems}
\begin{theorem*}
Division Algorithm \newline 
Let a,b be integers with $b > 0$. Then there $\exists q$ and r $\in \mathbb{Z}$ such that a = bq + r. where $0 \geq r < b$
\end{theorem*}
\begin{theorem*}
Fundamental Theorem of Arithmetic \newline 
$\forall n \in \mathbb{Z}$ except 0 is a product of primes
\end{theorem*}
\begin{theorem*}
Let $n > 1$. If n has no positive prime factors less than or equal to $\sqrt{n}$ then n is prime
\end{theorem*}
\subsection*{Practice Problems}
\begin{enumerate}
    \item Find the quotient q and remainder r when a is divided by b w/o the usage of technology
    \begin{itemize}
        \item[(a)] a = 17 b = 4
        \begin{equation*}
            17/4 = 4 R 1 
        \end{equation*}
        Therefore q = 4 and r = 1 then we have that 17 = 4(4) + 1 
        \item[(b)] a = -51 and b = 6 
        \begin{equation*}
            -51/6 = - 8 R 3 
        \end{equation*}
        Since - 8 is negative then q = -9 and r = 3 so therefore, 
        \begin{equation*}
            -51 = -9(6) + 3 
        \end{equation*}
    \end{itemize}
    \item Let a be any integer and let b and c be any integer divided by b, the quotient be q,and the remainder be r, so that 
    \begin{equation*}
        a = bq + r;  0 \geq r < b
    \end{equation*}
    Consider a, b, and c be any integer and 
    \item Find the GCD 
    \begin{itemize}
        \item[(a)] (56,72) 
        $$ 72 = 56(1) + 16$$
        $$ 56 = 16(3) + 8 $$
        $$ 16 = 8(2) $$
        Hence, the (56,72) = 8 
        \item[(b)] (143, 231) 
        $$ 231 = 143(1) + 88 $$
        $$ 143 = 88(1) + 55 $$
        $$ 55 = 33(1) + 22 $$
        $$ 33 = 22(1) + 11 $$
        $$ 22 = 11(2) $$
        (143,231) = 11
    \end{itemize}
   \item Express the numbers as a product of primes 
   \begin{itemize}
       \item[(a)] 5040 \newline 
       $$ 5040 = 2^4 \times 7 \times 9 $$
       \item[(b)] 
   \end{itemize}
   \item Which of the following are prime 
   \begin{itemize}
       \item[a)] $2^{5} - 1$
       \begin{equation*}
           2^5 - 1 = 32 - 1 = 31
       \end{equation*}
       Let n = 31 and let us take the square root of n then 
       \begin{equation*}
           \sqrt{31} \approx 5.6
       \end{equation*}
       There does not exist prime numbers less than 5 that 31 is divisible by, so therefore, 31 is prime. 
       \item[b)] 1951 \newline 
       The prime factorization of 1951 is $\pm 1, \pm 1951$ therefore, by definition of prime then 1951 is prime. 
       
   \end{itemize}
\end{enumerate}

\end{document}

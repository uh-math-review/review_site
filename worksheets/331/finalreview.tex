% DOCUMENT FORMATING
\documentclass[12pt]{article}
\usepackage[margin=1in]{geometry}

% PACKAGES
\usepackage{amsmath} % For extended formatting
\usepackage{amssymb} % For math symbols
\usepackage{amsthm} % For proof environment
\usepackage{array} % For tables
\usepackage{enumerate} % For lists
\usepackage{extramarks} % For headers and footers
\usepackage{blindtext}
\usepackage{fancyhdr} % For custom headers
\usepackage{graphicx} % For inserting images
\usepackage{multicol} % For multiple columns
\usepackage{verbatim} % For displaying code
\usepackage{tkz-euclide}
\usepackage{pgfplots}
\usepackage{algorithm}
\usepackage{algorithmic}
\usepackage[ruled,vlined]{algorithm2e}
% SET UP HEADER AND FOOTER
\pagestyle{fancy}
\lhead{\MyCourse} % Top left header
\chead{\MyTopicTitle} % Top center header
\rhead{\MyAssignment} % Top right header
\lfoot{\MyCampus} % Bottom left footer
\cfoot{} % Bottom center footer
\rfoot{\MySemester} % Bottom right footer
\renewcommand\headrulewidth{0.4pt} % Size of the header rule
\renewcommand\footrulewidth{0.4pt} % Size of the footer rule
% ----------
% TITLES AND NAMES 
% ----------

\newcommand{\MyCourse}{MATH 331}
\newcommand{\MyTopicTitle}{Final Exam Guide}
\newcommand{\MyAssignment}{Introduction to Real Analysis}
\newcommand{\MySemester}{Spring 2020}
\newcommand{\MyCampus}{University of Hawaii at Manoa}
\begin{document}
\subsection*{Key Terms}
\begin{itemize}
    \item Partition 
    \item Integrable
    \item Least Upper Bound Property 
    \item supremum 
    \item infrimum 
    \item sequence 
    \item convergence
    \item neighborhood 
    \item cauchy sequence 
    \item accumulation points 
    \item Bolzano Weiestrass Theorem 
    \item Sequential Limit Theorem 
    \item sub-sequence 
    \item monotone 
    \item increasing 
    \item decreasing 
    \item limits 
    \item continuity 
    \item uniform continuity 
    \item open 
    \item closed 
    \item compact 
    \item Heine Borel Theorem 
    \item Extreme Value Theorem 
    \item Bolzano's Theormem 
    \item connected 
    \item Intermediate Value Theorem 
    \item differentiable 
    \item Chain Rule 
    \item relative maximum 
    \item Rolle's Theorem 
    \item Mean Value Theorem 
    \item Cauchy Mean Value Theorem 
    \item L'Hospital's Rule 
    \item partition 
    \item integrable 
\end{itemize}
\subsubsection*{Sample Problems}
\begin{enumerate}
    \item Let the sequence ($a_n$) converge to A and $(b_n - a_n)$ converge to 0. Using the $\epsilon$ and N argument show that ($b_n$) converges to A. 
    \item Using $\epsilon-N$ argument prove that the sequence $(\frac{n}{2n+1})_{n=1}^{\infty}$ converges and find its limit.
    \item Define what it means for the sequence $\{a_n\}^\infty_{n=1}$ to converge to a real number $A$.
    \item Suppose f: [a,b] $\to \mathbb{R}$ is a bounded function. 
    \begin{itemize}
        \item Define what it means for P to be a partition of [a,b]
        \item Define a Lower Sum L(P,f)
        \item Define a Upper Sum U(P,f)
        \item Define the lower integral of f
        \item Define the upper integral of f 
        \item Define what it means for a function to be integrable 
    \end{itemize}
    \item State the following theorems 
    \begin{itemize}
        \item Mean Value Theorem
        \item Extreme Value Theorem
        \item Intermediate Value Theorem 
        \item Rolle's Theorem 
    \end{itemize}
    \item Give an example of an open cover of the set [1,5) that has no finite subcover 
    \item Define f: $\mathbb{R} \to \mathbb{R}$ by f(x) = $x^{2} - 5$. Use $\epsilon-\delta$ definition to prove that $\lim_{x \to 1} f(x) = -4$ 
    \item State one of the Theorems that gives a necessary and sufficient condition for f to be Riemann integrable on the interval [a,b].
    \item Suppose E $\subset \mathbb{R}$ is nonempty and that $E \cap [0,1] = \empty$
    \begin{itemize}
        \item Is it possible that the $\sup E = 0$
        \item Is it possible that the $\sup E = 1$
    \end{itemize}
    \item Suppose f:[0,2] $\to \mathbb{R}$ is defined by f(x) = 1 - $x^2$
    \begin{itemize}
        \item Explain how you can be sure that $f \in \mathbf{R}[0,2]$.
        \item For P the partition of [0,2] given by P = $\{0,0.5,1,2\}$ compute L(P,f)
    \end{itemize}
    \item Give the definitions of the following words 
    \begin{itemize}
        \item differentiable 
        \item uniformly continuous
        \item continuous 
        \item closed 
        \itme open 
        \item compact 
    \end{itemize}
    \item Prove that $g(x) = x
^3 + x - 1$ has at least one root which lies in the open interval
(0, 1).
    \item Prove that if f: $[a,b] \to \mathbb{R}$ is continuous, then f is Riemann integrable on [a,b]. 
    \item Prove that every Convergent Sequence is Cauchy 
    \item Answer these questions regarding to compact sets:
    \begin{itemize}
        \item State the Heine Borel Theorem 
        \item $\{-1,0,1\}$: Is it compact?
        \item $\{0\} \cup (1,4]$ Is it compact?
        \item $\{ \frac{1}{n}: n \in \mathbb{N}\}$ Is it compact?
    \end{itemize}
    \item The following Statement is false. Explain why. \newline 
    There is a function $f \in \mathbf{R}(x)$ on [-1,1] and a partition P of [-1,1] such that L(P,f) = 1 and U(P,f) = 2 and $\int_{-1}^{1} f(x) dx = 3$
    \item State True (T) or False (F) for the following: 
\begin{itemize}
    \item If A is a non-empty and compact set of real numbers then A contains inf A and sup A. 
    \item If f:(2,10) $\to \mathbb{R}$ is uniformly continuous, then it is bounded 
    \item If A and B are compact sets of real numbers then so is $A \cup B$
    \item If A and B are open sets of real numbers then so is $A \cup B$
    \item Every monotone sequence of real numbers converges.
    \end{itemize}
       \item Define a function f: $\mathbb{R} \to \mathbb{R}$
\[ \begin{cases} 
      0 & if x \in \mathbb{Q} \\
      x^{2} & if x \not \in \mathbb{Q} \\
   \end{cases}
\]
(a) Is f continuous at x = 0? Justify your answer. (Justification based on definition will receive the most points) 
\newline
(b) Is f differentiable at x = 0? Justify your answer.

\end{enumerate}
\end{document}

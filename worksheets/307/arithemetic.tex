% DOCUMENT FORMATING
\documentclass[12pt]{article}
\usepackage[margin=1in]{geometry}

% PACKAGES
\usepackage{amsmath} % For extended formatting
\usepackage{amssymb} % For math symbols
\usepackage{amsthm} % For proof environment
\usepackage{array} % For tables
\usepackage{enumerate} % For lists
\usepackage{extramarks} % For headers and footers
\usepackage{blindtext}
\usepackage{fancyhdr} % For custom headers
\usepackage{graphicx} % For inserting images
\usepackage{multicol} % For multiple columns
\usepackage{verbatim} % For displaying code
\usepackage{tkz-euclide}
\usepackage{pgfplots}
\newtheorem{theorem}{Theorem}[section]
\newtheorem*{theorem*}{Theorem}
\newtheorem{corollary}{Corollary}[theorem]
\newtheorem{lemma}[theorem]{Lemma}

% SET UP HEADER AND FOOTER
\pagestyle{fancy}
\lhead{\MyCourse} % Top left header
\chead{\MyTopicTitle} % Top center header
\rhead{\MyAssignment} % Top right header
\lfoot{\MyCampus} % Bottom left footer
\cfoot{} % Bottom center footer
\rfoot{\MySemester} % Bottom right footer
\renewcommand\headrulewidth{0.4pt} % Size of the header rule
\renewcommand\footrulewidth{0.4pt} % Size of the footer rule
% ----------
% TITLES AND NAMES 
% ----------

\newcommand{\MyCourse}{Math 307}
\newcommand{\MyTopicTitle}{Matrix Arithmetic }
\newcommand{\MyAssignment}{Linear Algebra and Diff. Eq}
\newcommand{\MySemester}{Fall 2020}
\newcommand{\MyCampus}{University of Hawaii at Manoa}

\begin{document}
\subsection*{Problems}
\begin{enumerate}
    \item Let A = $ \begin{bmatrix}
1 & 2 \\
0 & 4 \\
\end{bmatrix}$
B = $\begin{bmatrix}
2 & 1 \\
3 & 1 \\
\end{bmatrix}$
and C = $ \begin{bmatrix}
3 \\
5 \\
\end{bmatrix}$
Then if possible do these following operations 
\begin{itemize}
    \item[a)] A + B 
    \vspace{3cm}
    \item[b)] 3A - 2C
    \vspace{3cm}
    \item[c)] AC
    \vspace{3cm}
    \item[d)] CA 
    \vspace{3cm}
\end{itemize}
Perform the following matrix operation if possible.
\item $ \begin{bmatrix} 
2 & 0 & 0 \\
0 & 1 & 0 \\
0 & 0 & 4 \\
\end{bmatrix}
 \begin{bmatrix} 
1 & 0 & 0 \\
0 & 2 & 0 \\
0 & 0 & 1 \\
\end{bmatrix} = $
\clearpage
\item $
\begin{bmatrix}
2 & 4 \\
0 & -1 \\
\end{bmatrix}
\begin{bmatrix}
2 & 0 \\
1 & -1 \\
\end{bmatrix}
 = $
 \vspace{3cm}
 \item $
\begin{bmatrix}
2 & 4 \\
0 & -1 \\
\end{bmatrix}
+ \begin{bmatrix}
1 & 0 & 0 \\
3 & 1 & 0 \\
\end{bmatrix}
 = $
 \vspace{3cm}
 
\end{enumerate}

\end{document}
